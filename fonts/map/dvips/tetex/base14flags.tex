% $Id: base14flags.tex 31831 2013-10-04 18:33:49Z karl $
% Copyright 2013 Karl Berry.
% You may freely use, modify, and/or distribute this file.
% 
% Use each of the base14 fonts, explicitly eliminating fontflags from
% the map entries.  This causes pdftex to internally generate them
% (mapfile.c) and issue a warning including the value, so we can see
% what they are.  (See ./Makefile for a little rule to extract the values.)
%
% Why do we go to the trouble of using flag values for the base14 fonts,
% instead of eliminating the flag warning altogether in the source?
% Because it's conceivable (though not practical) that someone would
% want to specify some other font as not-downloaded, and hence want to
% specify flags for it.  In general, it just seems safer to specify the
% flags where they should be specified instead of ignoring the issue.

\nopagenumbers
\def\dofont#1#2{%
  \pdfmapline{#1 #2 <8r.enc}%
  \font\0 = #1 \0 \hbox{#1 #2}\space
}

% base14 only, not base35.
\dofont{pcrb8r}{Courier-Bold}
\dofont{pcrbo8r}{Courier-BoldOblique}
\dofont{pcrr8r}{Courier}
\dofont{pcrro8r}{Courier-Oblique}
%
\dofont{phvb8r}{Helvetica-Bold}
\dofont{phvbo8r}{Helvetica-BoldOblique}
\dofont{phvr8r}{Helvetica}
\dofont{phvro8r}{Helvetica-Oblique}
%
\dofont{psyr}{Symbol}
%
\dofont{ptmb8r}{Times-Bold}
\dofont{ptmbi8r}{Times-BoldItalic}
\dofont{ptmr8r}{Times-Roman}
\dofont{ptmri8r}{Times-Italic}
%
\dofont{pzdr}{ZapfDingbats}

\end
